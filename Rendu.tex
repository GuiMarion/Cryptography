% !TEX encoding = UTF-8 Unicode


\documentclass[10pt]{article}
\usepackage{geometry}                % See geometry.pdf to learn the layout options. There are lots.
\geometry{letterpaper}                   % ... or a4paper or a5paper or ... 
%\geometry{landscape}                % Activate for for rotated page geometry
%\usepackage[parfill]{parskip}    % Activate to begin paragraphs with an empty line rather than an indent
\usepackage{graphicx}
\usepackage{amssymb}
\usepackage{epstopdf}
\usepackage[utf8]{inputenc}  
\usepackage[T1]{fontenc}      
\usepackage[francais]{babel}  
\usepackage[utf8x]{inputenc}


\title{Rendu Cryptographie}
\author{Guilhem Marion}
\date{}                                           % Activate to display a given date or no date

\begin{document}
\maketitle


\section*{Exercice 1}

\subsection*{Question 1}

Si deux individus ont la même $N$ ainsi que deux clefs publiques/privées differentes c'est qu'ils connaissent $\phi(N)$\footnote{Le nombre de nombres entiers en 1 et N.}. Or, 

\begin{center}

$e_1d_1 = 1  (mod \phi(N)$)

\end{center}

et 

\begin{center}

$e_2d_2 = 1   (mod \phi(N)$)

\end{center}

Et $e_1$ et $e_2$ sont publics et $\phi(N)$ et connu des deux individus, donc on peut retrouver $d_1$ (respectivement $d_1$) en calculant l'inverse de la clef publique de l'autre individu modulo $\phi(N)$.


\end{document}  











